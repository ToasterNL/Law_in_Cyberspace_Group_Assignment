\documentclass[11pt]{article}
\usepackage{a4wide, graphicx, fancyhdr, wrapfig, tabularx, amsmath, amssymb, hyperref, color, verbatim, nameref}
\usepackage[english]{babel}
\definecolor{linkcolour}{rgb}{0,0.2,0.6}
\hypersetup{colorlinks,breaklinks,urlcolor=linkcolour, linkcolor=linkcolour}

%----------------------- Macros and Definitions --------------------------

\setlength\headheight{20pt}\usepackage{}
\addtolength\topmargin{-10pt}
%\addtolength\footskip{20pt}

\fancypagestyle{plain}{%
\fancyhf{}
\fancyfoot[RO,LE]{\sffamily\bfseries\thepage}
\renewcommand{\headrulewidth}{0pt}
\renewcommand{\footrulewidth}{0pt}
}

\pagestyle{fancy}
\fancyhf{}
\fancyfoot[RO,LE]{\sffamily\bfseries\thepage}
\fancyhead[RO,LE]{\textsc{team Pooh Bear}}
\fancyhead[LO,RE]{\emph{Setup description}}
\renewcommand{\headrulewidth}{1pt}
\renewcommand{\footrulewidth}{0pt}
\newcommand{\tab}{\hspace*{2em}}

\newcommand{\tocheck}[1]{{\bf !?: #1 :!?}}
\frenchspacing

%-------------------------------- Title ----------------------------------

\title{\textbf{Law in Cyberspace\\ \emph{Behavioural Advertising}}}
\author{
	H.J. Tilmans
	\and M. Vijfvinkel
	\and Y. Zeng
	\and
	\href{mailto:h.j.tilmans@student.tue.nl}{h.j.tilmans}@student.tue.nl,\\
	\{\href{mailto:m.vijfvinkel@student.ru.nl}{m.vijfvinkel},
	\href{mailto:y.zeg@student.ru.nl}{y.zeng}\}@student.ru.nl
}
\date{\today}

\begin{document}
\maketitle
\section{Introduction}
The online advertising market topped a record revenue of 26 billion US Dollar over the year 2010. \cite{IAB2011}
That given, it is not surprising that the advertising business is constantly looking for methods to improve ad responses and thus revenue.
Nowadays, online behavioural advertising - a technique that tracks online user's behaviour - is the main method used for delivering advertisements to each user's individual estimated interest. Estimated, because the delivered ads are chosen from a large database (containing advertisers' ads) according to the digital profile that is created for each online user. Clearly, that profile may not correspond with the real user's 'profile'.
Moreover, what exactly does the advertising world know about you?

In the next section, a short overview is given on the techniques used for online behavioural advertising. In section~\nameref{legal violations}, the privacy violations \tocheck{and data protection violations} are defined. Then the current applicable laws from the EU and the European Convention of Human Rights are stated, followed by the section~\nameref{memorandum} in which advises to the courts, legislators and industries are given.
\section{Online Behavioural Advertising Techniques}
\label{techniques}

\section{The Legal Violations}
\label{legal violations}

\section{Applicable Laws}
\label{applicable laws}

\section{Memorandum}
\label{memorandum}
\subsection{To The Courts}
\subsection{To The Legislators}
\subsection{To The Industry}

\section{Conclusion}

%%% REFS %%%

\bibliographystyle{plain} % amsalpha
\bibliography{assignment2}

\end{document}
