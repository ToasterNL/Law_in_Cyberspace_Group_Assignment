\documentclass[11pt]{article}
\usepackage{a4wide, graphicx, fancyhdr, wrapfig, tabularx, amsmath, amssymb, hyperref, color, verbatim, nameref}
\usepackage[english]{babel}
\definecolor{linkcolour}{rgb}{0,0.2,0.6}
\hypersetup{colorlinks,breaklinks,urlcolor=linkcolour, linkcolor=linkcolour}

%----------------------- Macros and Definitions --------------------------

\setlength\headheight{20pt}\usepackage{}
\addtolength\topmargin{-10pt}
%\addtolength\footskip{20pt}

\fancypagestyle{plain}{%
\fancyhf{}
\fancyfoot[RO,LE]{\sffamily\bfseries\thepage}
\renewcommand{\headrulewidth}{0pt}
\renewcommand{\footrulewidth}{0pt}
}

\pagestyle{fancy}
\fancyhf{}
\fancyfoot[RO,LE]{\sffamily\bfseries\thepage}
\fancyhead[RO,LE]{\textsc{}}
\fancyhead[LO,RE]{\emph{}}
\renewcommand{\headrulewidth}{1pt}
\renewcommand{\footrulewidth}{0pt}
\newcommand{\tab}{\hspace*{2em}}

\newcommand{\tocheck}[1]{{\bf !?: #1 :!?}}
\newcommand{\OBA}{Online behavioural advertising }
\newcommand{\oba}{online behavioural advertising }
\newcommand{\ePD}{ePrivacy Directive }
\newcommand{\DPD}{Data Protection Directive }

\frenchspacing

%-------------------------------- Title ----------------------------------

\title{\textbf{Law in Cyberspace\\ \emph{Web Analytics and Online Personalisation}}}
\author{
	H.J. Tilmans
	\and M. Vijfvinkel
	\and Y. Zeng
	\and
	\href{mailto:h.j.tilmans@student.tue.nl}{h.j.tilmans}@student.tue.nl,\\
	\{\href{mailto:m.vijfvinkel@student.ru.nl}{m.vijfvinkel},
	\href{mailto:y.zeg@student.ru.nl}{y.zeng}\}@student.ru.nl
}
\date{\today}

\begin{document}
\maketitle


\section{Introduction}
The online advertising market topped a record revenue of 26 billion US Dollar over the year 2010. \cite{IAB2011}
That given, it is not surprising that the advertising business is constantly looking for methods to improve ad responses and thus revenue.

Nowadays, \oba - a technique that tracks online user's behaviour - is the main method used for delivering advertisements to each user's individual estimated interest. Estimated, because based on the traced user's online activities, an algorithm will create/update the corresponding user profile. Unfortunately, wrong conclusions may be taken by the algorithm, which may lead to having a user profile that does not correspond to the user's real 'profile'.

But what implications does this have for a user? Moreover, what exactly does the advertising world know about you?
Note that in this memorandum the focus will be on European legislation.

\section{Terminology}
\label{terminology}
First we will provide some terminology and explain how an \oba system works.
In a typical \oba system the following four participating entities can be distinguished:

\begin{itemize}
	\item Ad network\\
			The ad network makes money by charging the advertisers who would like to use the big advertising infrastructure of the ad networks (e.g. Google's Adsense). With the use of online behavioural tracking, the ad networks can deliver ads to online users more accurately.

	\item Advertiser\\
			The advertisers use ad networks for spreading their ads more efficiently. Meaning, when using ad networks, the advertiser's ads most likely will be shown to interested users more than to uninterested users, which clearly should result in more revenue for the advertisers.

	\item Ad publisher\\
			The ad publisher shows the targeted ads to the online users through their website. In return for the used online web-space, the ad publisher receives money per click/view.

	\item Online user\\
			Browsing through the behavioural advertising supported websites, the user will receive more interesting, thus less annoying advertisements. Furthermore, thanks to the financial support of the \oba to publishers, the user can enjoy many more free online services, e.g. Google Mail. 
\end{itemize}

Clearly, everyone benefits from this \oba system.
When the ad networks use better tracking techniques, they will be able to target the ads better, which will have more users view or click on interesting ads and thus having the ad publishers earning more money, as well as the advertisers who will have more potential buyers on their website.
So, which techniques are used to accomplish this?

\section{Facts}
In this section we will describe which techniques are used in \oba market and which European legislation applies.

\subsection{\OBA tracking techniques}
\label{techniques}
Online behavioural advertising completely relies on the creation of accurate user profile(s) by the ad networks. Whenever a user profile is incomplete or incorrect, the user will receive ads which he may not like. To avoid this, ad networks use, inter alia, the following techniques.

\begin{itemize}
	\item Cookies\\
		A cookie is a piece of text received from the web server - when a user is visiting a website - and is saved on his computer by the web browser. Aside from a unique user identifier, site preferences, authentication information, session identifiers, more information can be stored in this cookie as well. Whenever the user is visiting the same website again, this cookie is sent to the website which will adjust the content of the website according to your preferences and profile. Google Adsense, for example, stores a cookie with a unique number in order to be able to track you when visiting ad publisher's websites (which could be connected to Google Adsense's network). \cite{Adsense2011}

	\item Web Bugs\\
		A web bug is usually a small, seemingly invisible, part of a web page or non-plain text e-mail, which allows the ad network to track who has viewed the web page or e-mail. This web bug is downloaded from the ad networks server every time the web page or e-mail is opened, so the ad networks know when and what you opened. Possibly, JavaScript is used to extract even more information from you- e.g. by sniffing browser history - to ensure the ad networks use the right profile. As this technique can completely run on only JavaScript, no cookies are required.

	\item Web browser fingerprinting\\
		A web browser fingerprint is the collection of all information about the browser version, operating system it runs on, installed plug-ins and their version numbers and more. In a publication of the Electronic Frontier Foundation (EFF), a web browser can already be uniquely identifiable among over 500,000 (privacy conscious) users. \cite{EFF2010} Like with the web bugs, the web browser fingerprinting does not require client-side storage. But, not all information can be extracted from the web browser without the use of JavaScript.

	\item Keystroke statistical analysis\\
		When knowing keystroke speeds or intervals of a user, one could identify a user according to their 'keystroke profile'. This technique is based on the fact that each user has a so-called unique keystroke 'fingerprint'. Google may be using the technique already, since whenever you have \textit{Google Instant} \cite{GoogleInstant} activated, Google receives every character you type in the search box separately. This technique requires JavaScript. 
\end{itemize}

As can be seen, lots of information can be gathered using these techniques. But, what privacy implications do these techniques introduce? Is it allowed to create digital profiles of users, without the user knowing? The legal implications and violations will be discussed in the next section.

%\subsection{OBA policies}
%<Mark>: Ik neem aan dat je hier de policies van Google Adsense of Facebook bedoeld? Misschien is dat wat overbodig omdat hier te quoten en kunnen we het beter weglaten en misschien her en der in de tekst naar verwijzen. Als je iets totaal anders bedoeld...wat bedoel je dan precies?

\subsection{Applicable European legislation}
Since we describe this problem from a European point of view, we also have to look at the European legislation that deals with this problem. Below we will quote relevant articles from the \ePD, the European Convention of Human Rights and the \DPD and the European Cookie Directive \cite{cookielaw}.
%\tocheck{Hier kan het stuk van Sherry, betreffende de wetten!}

%http://eur-lex.europa.eu/LexUriServ/LexUriServ.do?uri=OJ:L:2009:337:0011:0036:En:PDF     cookie directive, staat als het goed is ook op git/resources

\subsubsection{\ePD 2002/58/EC}

Out of the \ePD we use the following articles:

\begin{itemize}
	\item Definition Electronic Mail\\
		\emph{ ‘electronic mail’ means any text, voice, sound or image message sent over a public communications network which can be stored in the network or in the recipient's terminal equipment until it is collected by the recipient;}

	\item Article 3\\
		\emph{This Directive shall apply to the processing of personal data in connection with the provision of publicly available electronic communications services in public communications networks in the Community, including public communications networks supporting data collection and identification devices.}

	\item Article 6\\
		\emph{For the purpose of marketing electronic communications services or for the provision of value added services, the provider of a publicly available electronic communications service may process the data referred to in paragraph 1 to the extent and for the duration necessary for such services or marketing, if the subscriber or user to whom the data relate has given his or her prior consent. Users or subscribers shall be given the possibility to withdraw their consent for the processing of traffic data at any time.}

	\item Article 12\\
		\emph{Member States shall ensure that subscribers are informed, free of charge and before they are included in the directory, about the purpose (s) of a printed or electronic directory of subscribers available to the public or obtainable through directory enquiry services, in which their personal data can be included and of any further usage possibilities based on search functions embedded in electronic versions of the directory.}

	\item Article 13\\
		\emph{The use of automated calling and communication systems without human intervention (automatic calling machines), facsimile machines (fax) or electronic mail for the purposes of direct marketing may be allowed only in respect of subscribers or users who have given their prior consent.}

		\emph{Member States shall take appropriate measures to ensure that unsolicited communications for the purposes of direct marketing, in cases other than those referred to in paragraphs 1 and 2, are not allowed either without the consent of the subscribers or users concerned or in respect of subscribers or users who do not wish to receive these communications, the choice between these options to be determined by national legislation, taking into account that both options must be free of charge for the subscriber or user.}
%\section{Survey of pertinent statues}   %optional
\end{itemize}

\subsubsection{Legal implications of \ePD}

The articles mentioned above do not show all the subarticles, but show the main points related to this problem. The \ePD shows that if electronic communication services and public communications networks want to process personal data, they have to ask prior consent from the user to do so. The \ePD also shows that unsolicited communications used for direct marketing needs to have prior consent of users, or national legislation has to make sure that the users wishes not to receive the communication, are respected.

\subsubsection{European Convention of Human Rights}
\begin{itemize}
	\item Article 8 - Right to respect for private and family life
		\begin{itemize}
			\item [\textit{1.}] {\it Everyone has the right to respect for his private and family life, his home and his correspondence.}
			\item [\textit{2.}] {\it There shall be no interference by a public authority with the exercise of this right except such as is in accordance with the law and is necessary in a democratic society in the interests of national security, public safety or the economic well-being of the country, for the prevention of disorder or crime, for the protection of health or morals, or for the protection of the rights and freedoms of others.}
		\end{itemize}
	\item Protocol No. 12, Article 1 - General prohibition of discrimination 
		\begin{itemize}
			\item [\textit{1.}] {\it The enjoyment of any right set forth by law shall be secured without discrimination on any ground such as sex, race, colour, language, religion, political or other opinion, national or social origin, association with a national minority, property, birth or other status.}
			\item [\textit{2.}] {\it No one shall be discriminated against by any public authority on any ground such as those mentioned in paragraph 1.}
		\end{itemize}
\end{itemize}


\subsubsection{Legal implications of European Convention of Human Rights}
As becomes clear from \textit{article 8}, everyone's privacy has to be respected, unless if that person is a legal risk factor. This is - sort of - regulated through clear consent that a user has to give, before privacy related information may be gathered.
However, it is impossible for everyone to check whether or not he is being discriminated. For example, the OBA mechanism could discriminate user $x$ and $y$ by returning different pricing to both the users, based on their online spending behaviour. When user $x$ generally is willing to pay more than €100 for a watch, he may receive higher pricing for sunglasses than user $y$, who, for example, is not willing to pay more than €50 for watches. This sounds extreme, but who can tell if these techniques are not already is use?

\subsubsection{\DPD 95/46/EC}
\label{DPD}
\begin{itemize}
	

	\item Article 7\\
		\emph{Member States shall provide that personal data may be processed only if:}
			\begin{itemize}
				\item  [\textit{a.}] {\it the data subject has unambiguously given his consent; or}
				\item  [\textit{b.}] {\it processing is necessary for the performance of a contract to which the data subject is party or in order to take steps at the request of the data subject prior to entering into a contract; or}
				\item  [\textit{c.}] {\it processing is necessary for compliance with a legal obligation to which the controller is subject; or}
				\item  [\textit{d.}] {\it processing is necessary in order to protect the vital interests of the data subject; or}
				\item  [\textit{e.}] {\it processing is necessary for the performance of a task carried out in the public interest or in the exercise of official authority vested in the controller or in a third party to whom the data are disclosed; or}
				\item  [\textit{f.}] {\it processing is necessary for the purposes of the legitimate interests pursued by the controller or by the third party or parties to whom the data are disclosed, except where such interests are overridden by the interests for fundamental rights and freedoms of the data subject which require protection under Article 1 (1).}
			\end{itemize}

	\item Article 8 - The processing of special categories of data\\
		\begin{itemize}
			\item  [\textit{1.}] {\it Member States shall prohibit the processing of personal data revealing racial or ethnic origin, political opinions, religious or philosophical beliefs, trade-union membership, and the processing of data concerning health or sex life.}
			\item  [\textit{2.}] {\it Paragraph 1 shall not apply where:}
			\begin{itemize}
				\item  [\textit{a.}] {\it the data subject has given his explicit consent to the processing of those data, except where the laws of the Member State provide that the prohibition referred to in paragraph 1 may not be lifted by the data subject's giving his consent; or}		
			\end{itemize}

		\end{itemize}

	\item Article 14 - The data subject's right to object\\
		\emph{Member States shall grant the data subject the right:}
		\begin{itemize}
			\item  [\textit{b.}] {\it to object, on request and free of charge, to the processing of personal data relating to him which the controller anticipates being processed for the purposes of direct marketing, or to be informed before personal data are disclosed for the first time to third parties or used on their behalf for the purposes of direct marketing, and to be expressly offered the right to object free of charge to such disclosures or uses.}
		\end{itemize}

	\item Limited collection principle (art. 10 and 11)\\
The collection of personal information shall not be arbitrary. Before collecting personal information, the data subjects should be informed of the collected purpose, scope of collection, use and so on. Besides, the data subject should also be informed that information may be disclosed or provided to any person, in order to give option to information subjects to decide whether to provide their personal information, and in what form and way to provide their personal data. If the information is not directly received from the data subject, information subjects should be directly informed what information is collected, and information content, in order to guarantee citizens should enjoy the right to know.

	\item Legal dispose principle (art. 6)\\
Processing of personal data, should get the consent from the data subjects, otherwise, the processing is limited in some circumstances. These limited circumstances include: the purpose of fulfilling the contracts, or to deal with situations such as personal information in order to protect data subjects’ great profit.

	\item Personal participate principle (art.14)\\
Personal data subject should have reasonable control over the data, means data subjects should be reasonable entitled to access their individual information. They have the right to request the data controller to correct, add, and delete personal data, to ensure personal information accurate and complete.

	\item Proper manage principle (art. 22)\\
Data controller and consumer shall commitment for the safety of personal data. Storage of personal information should take a scientific approach, using technology. They should manage the operation measures, to prevent data loss, damage, alteration, loss, improper distribution, handling and so on. In addition, they should also take reasonable security measures to prevent these hazards.
		

\end{itemize}

\subsubsection{Legal implications of the \DPD}
The \DPD clearly states that the user has to unambiguously give his consent before his personal data may be processed, and explicit consent is needed when it comes to information described in \textit{article 8.1}. A user also has the right to object to processing of personal data if they are being used for direct marketing, even by third parties.

The principles above build the basic structure for how to identify the invading personal data. Once these principles are violated, corresponding punishment is enforced to the offender.

\subsection{Conclusion legal implications}
We can conclude from the above mentioned directives that the gathering and processing of user data is not allowed, unless the user gives his consent. This results in legal friction when we look at the current techniques used in behavioural advertising as mentioned in the introduction. For the average user, it is difficult to determine if data is being collected via these techniques in first place.  The cookie law \cite{cookielaw} for example, does not prohibit the use of web bugs because they do not write or read information from the user's computer. We will discuss the legal issues in the next chapter using a case study about the social networking site Facebook and other cases. We will discuss the legal issues in the next chapter using a case study about the social networking site Facebook and a mechanism for implementation and effectiveness.


\section{Legal issues according case studies}
In order to explain the legal issues we will first introduce some questions about the legality of \oba industry. These questions will be answered using a case that is in progress, Europe versus Facebook and a mechanism for implementation and effectiveness.

\subsection{Legal issues unanswered}
\label{sec: legal questions}
Because the effectiveness of \oba strongly depends on harvesting personal user information, the following legal issues are introduced:\\

\begin{itemize}
	\item \textit{Is the OBA industry allowed to gather the personal information the way they do nowadays?}\\
	\item \textit{Is the OBA industry allowed to process the personal information the way they do nowadays}\\
	\item \textit{Is the OBA industry unlawfully discriminating users based on their profile?}\\
%	\item \textit{}\\
%	\item \textit{}\\
\end{itemize}

\noindent Below we will try to answer these questions by introducing one case study and mention some mechanisms for implementation and effectiveness .

\subsection{Europe vs. Facebook}
\label{sec: EU vs FB}
On the 18th of August 2011 sixteen complaints were filed against Facebook Ireland Limited, asking the Irish Data Protection Commissioner (DPC) to investigate these complaints. These complaints are part of one big complaint, but have been split up to make it easier for the DPC. The formal complaint states that Facebook Ireland, on a European level, breaks the Data Protection Directive.\footnote{In the complaints on the website they refer to Directive 94/46/EC, we asked for an explanation of why they chose this Directive. It turned out to be a typo and is a matter they have to look into. We were told that the complaints are informal so typos would not matter. Eventually the typo was fixed.}
On the 24th of August 2011 the DPC replied with a statement that they will investigate the complaints. On the 15th of September a letter was send to the DPC, stating that Facebook changed its privacy policy and settings. Revisions of some complaints were made although they still complied with a breach of the Data Protection Directive. On the 15th of September another six complaints were against Facebook.

All these complaints had one general theme in common and state the following:

\begin{itemize}

\item Facebook users agree to terms of Facebook. In section 18.1 the terms state, that users living outside of the U.S. or Canada have a contract with Facebook Ireland.
\item To preform the contract Facebook Ireland processes personal data in different ways. Therefore Facebook Ireland is the controller and the Irish Data Protection Act applies.
\item The complaints make a distinction between the hosting of data and the further processing of data. Hosting means that the user is the controller and Facebook Ireland is the processor. In the case of further processing for Facebook Ireland's own purpose, Facebook Ireland is regarded as the sole controller.
\item Because also the hosted data is used by Facebook Ireland for its own purpose, it must always be seen as the controller. 
\item If Facebook Ireland still processes data that was "removed" by the user, then the user is not in control of the data and Facebook Ireland is the sole controller.
\item Several complaints will be filed against Facebook Ireland in order to make a distinction between the different types of data that Facebook Ireland processes.

\end{itemize}

As this is still an ongoing case (the audit has been completed, but the results have not been made public yet), it is not possible to predict the outcome. We can determine that there is some legal ground for the processing of data which is removed by the user. Facebook's worldwide revenue depends on the processing of personal data, which again can be used for advertisement. The outcome of this case is therefore very interesting for the behavioural advertisement market. If it turns out that Facebook Ireland does indeed break the Irish Data Protection Act then this can be applied to the \DPD, as the Irish Data Protection Act is based on this. Since all EU member states must comply with the \DPD, Facebook would break the Data Protection Directive in every member state. This would mean that other companies in the behavioural advertisement industry could expect similar complaints and investigations. If the outcome is in favour of Facebook Ireland, the behavioural advertisement industry will get more leverage in future cases, which might result in more hefty discussion about personal data and privacy.


\subsection{Mechanisms for implementation and effectiveness}

Violation of the European data protection law will result in significant enforcement actions, but for some reasons, it is no common to impose the penalty. As the European Commission and other EU institutions do not have enforcement jurisdiction, so enforcement action always be implemented according to local law by domestic courts or data protection agencies.

\begin{itemize}
	
\item Inspection and investigation for data processing facilities (\DPD art.28)\\
After receiving the relevant complaints or inspection, the data protection agency has inspection rights for the company's facilities. In 2000,150 complaints were filed with the Portuguese DPA, and the Authority carried out 135 inspections, including an audit of a tele-communications operator \cite{MSIberica}.

\item Penalty
The amount may be huge: On convictions on indictment, the maximum penalty is a fine of €100,000 \cite{offenceDPact}.
A prime example is the fine of 10 million pesetass (approx €60,000) levied in July 2000 by the Spanish DPA on Microsoft Iberica SRL for processing personal data without the consent of data subjects \cite{MSIberica} Section 1.65.

\item Criminal sanctions (usually only applies particularly significant violations)
Refuse or lack of cooperation with data protection authorities, which may result in criminal sanctions.

\end{itemize}




\subsection{Legal issues answered}
We will now provide answers to the questions asked in section \ref{sec: legal questions}.

\begin{itemize}
	\item \textit{Is the OBA industry allowed to gather the personal information the way they do nowadays?}\\
Whenever the OBA businesses complies with the cookie law \cite{cookielaw}, they are legally allowed to gather the personal information. Nevertheless, as we have discussed, cookies are just one of the many tracking techniques. Since the other tracking techniques are not encapsulated by the current cookie law, the OBA industry is not legally prohibited to (surreptitiously) track users using those techniques. 

As Google is the most used search engine - Google has \textit{global rank 1} \cite{googlerank1} on the well-established website \texttt{alexa.com} - they have a vast establishment on which they can apply other tracking techniques than cookies. One example would be the statistical analysis of user's keystroke when they enter the search keywords in Google. This analysis on its own can already to uniquely identify a single online user \cite{keystroke}, which leaves out the need to having cookies.

Because we can not check if the OBA industry is using those 'rogue' techniques currently, laws should be created and enforced to help users to protect against this.

	\item \textit{Is the OBA industry allowed to process the personal information the way they do nowadays?}\\
In section \ref{sec: EU vs FB} we established that there is some legal ground for the processing of data by Facebook. The complaints state that Facebook is processing the personal data of its users in a way, that user did not give his consent for. As the complaints also state is this a violation of the \DPD. Unfortunately this case is still ongoing and cannot be used as case law, but it does provide a future reference and will hopefully provide an answer about this matter.
			
	\item \textit{Is the OBA industry unlawfully discriminating users based on their profile?}\\
In a real shop, every shopper will pay the same amount for the same products or services. In online shops, this is not possible to check at this moment. Pricing may be altered on-the-fly, using a user's online profile. Rumours circulate the internet that this is happening already, but this has neither been confirmed or debunked. According to \texttt{timetowander.com/how-the-cookie-crumbles/}, a user of Ryanair found that the prices of a certain flight went up the next, when he came back to check out the same flight. After clearing the cookies the prices went down again. 

% Since the OBA business' ad selection algorithms are not know, this can not be checked.

\end{itemize}





%\section{Survey of precedents}
%"You must review the relevant, primary precedents governing the facts. It is usually not
%necessary to prepare a history of the case law; the most recent or definitive cases will
%suffice."

%\section{Discussion of each issue}
%"A dispassionate discussion of the issues and the applicable law is the central purpose of
%the memorandum. In this section you predict the answers that a court would give if it
%were faced with your facts, given the pertinent law.
%"
%\begin{itemize}
%	\item \textit{Is the OBA industry allowed to gather the personal information the way they do currently?}\\
			


%	\item \textit{Is the OBA industry allowed to use the personal information the way they do currently}\\
%				They use it for better ad targeting. But do they forward the information to third parties as well? Can not be checked, should be able to check or be punishable. \tocheck{nog dingen vertellen}

%	\item \textit{Is the OBA industry unlawfully discriminating users based on their profile?}\\
%			 \tocheck{nog dingen vertellen}

%	\item \textit{}\\
%	\item \textit{}\\
% \end{itemize}



\section{Conclusion}
So far, we have made a step into the world of web analytics and online personalisation, and discussed the corresponding legal issues that are introduced. As we have discussed some prime case laws, giving an overview on the current legal capabilities of the individuals, industries and courts, it became clear that current legislation is insufficient in resolving the privacy and data protection issues that come along within this world. Therefore, we will give some recommendations to the courts to handle related cases with current legislation, give advices to the legislators to improve the effectiveness of the desired goals of these laws and last but not least, give advice to the industry as well.


\section{Recommendation}
\subsection{To the courts}

As can be concluded from section \ref{DPD}.

National or local DPAs at the Member State level typically lack the resources and personnel to engage in widespread enforcement activities. The rapid growth of the internet has made it difficult for them to monitor compliance with data protection rules in the electronic commerce sector \cite{MSIberica} Section 1.64. Such unsubstantial enforcement encourages invading activities to some degree. Therefore, the advice to court is that they should swift their attention from keep creating new laws to strengthen penalty measures, so as to frighten lawbreaker, fundamentally decrease invading behaviors incidence.
\newline
\newline
Other advice that should be given to the court: 

The courts should look at how the information is question was collected, especially if the consent of the user was asked or not. It should be taken into account that most users have no idea that they are being tracked and that most people skip reading of the
terms of the website in question. Users also blindly accept these terms in order to use the services that are provided. 

\subsection{To the legislators}
As explained in the previous sections, current legislation is not accurate or sufficient enough to solve the legal issues discussed in this memorandum. In order to verdict similar cases in the same way, we would like to advice the legislators in editing current laws and in creating new laws, regarding this topic. Below, several recommendations/advices are given.

%het adviseren dat de wetgever probeert te anticiperen op de ontwikkelingen in de technologie. Nu is het zo dat de wet achter de techniek aanloopt en dat zodra er uiteindelijk nieuwe wetgeveing is, er een nieuwe techniek is die niet door die wet gevangen kan worden.

\begin{itemize}
	\item Online behavioural advertising not only depends on cookies, as can be seen our section on 'OBA techiques'. A more general approach for the defense against OBA should be used in our opinion. The cookie law \cite{cookielaw} should be rewritten or a new directive should be introduced which accomplishes the same as this cookie law but encapsulates the principle of user tracking instead of focusing on only one technique.
	\item We would also like to see more involvement of the legislators with current technology. They should try to anticipate what the impact of new technologies can be and adapt the legislation accordingly.
	\item The sanctions that are linked to the breaking of these laws, should be increased. The amount of the fines are at this moment to low and should be raised.
\end{itemize}

\subsection{To the industry}
It may be clear that the online user's privacy is of the main (legal) concern. However, in order to sustain the ad financed online services, so-called privacy enhanced technologies (PETs) are required. In the ideal case, a PET will provide maximum privacy protection, but will not limit the business' revenue. Taking this as the ultimate goal, we can derive the following requirements:

\begin{itemize}
	\item Integration\\
		The PET should be easily integrable with the current system. Having a system that requires major changes to the current system will withhold the industry from implementing PETs. Unless, of course, having installed PETs is legally obliged.

	\item Scalability\\
		The PET should be scalable, which means that it should not introduce (exponential) amounts of processing time or increase bandwidth drastically. If it would do so - i.e. it will take more than 1 minute before receiving the query results - user convenience and experience will decrease drastically. This again will have a negative effect on the effectiveness of the ads, which is not the desired effect of a PET.

	\item Trusted third parties (TTPs)
		The PET should create technological guarantees of preserving user's privacy. Since the users probably will not trust the advertising businesses, TTPs will be necessary to create these technological guarantees. But, to exclude the possibility of having a collusion between a TTP and the advertising market, independent authorities should monitor and audit the advertising market and the TTPs.

	\item Vulnerabilities of the PET and their impact in case of an attack
		The PET should have an as little as possible attack surface, and the existing attacks should have as small as possible negative impact on the users privacy or the advertising market's revenue. Obviously, having users that can work-around the ads will cause less revenue for the advertising market, and having the advertising market work-around the privacy preserving measures will violate the user's privacy.

	\item Evidence preserving
		In case of unlawful actions from any of the involved parties (e.g. privacy violations by the advertising market), clear, trusted and undeniable evidence of such actions should be present. 
\end{itemize}

Other ways the industry can help to solve this problem, is by making an opt-in system standard instead of the current opt-out system. This way users do not have to go through  different settings in order to switch parts of the \oba system off. 
The industry can also offer a basic service for free that does not track the user and can offer an extended service that enables tracking, but has more options. This has to be made clear to the user of course.


%\section{Introduction}
%In the next section, the terminology used in this memorandum is explained. Then, in section~\nameref{techniques}, a short overview is given on the techniques used for \oba. In section~\nameref{legal implications}, the privacy violations \tocheck{and data protection violations} are defined. Then the current applicable laws from the EU and the European Convention of Human Rights are stated, followed by the section~\nameref{advices} in which advises to the courts, legislators and industries are given.

%\section{The legal implications}
%\label{legal implications}


%---------------------------------------------

% So far I have read about half of the e-privacy directive and marked some text, that might be relevant. This is not the text with the actual articles yet, but the introductory text, marked by (number) on pages 1-10.

% (16), (17), (22), (24), (25), (26), the rest i still have to read.
% Another EU directive, is de data protection act (i believe it is 95/46/EC and correct if i'm wrong)...the eprivacy directive refers to that one.
% Directive 2009/136/EC refers to cookies in particular (dutch cookie law is based on this), which might be interesting to mention a way that law slows the advertisers down. We can use this as an example of something that is mostly ignored by websites so far/ might cause inconvience for users (accepting cookies al the time) and reduce revenue for advertisers. 
%From there we can build to the Europe vs Facebook case, of which I don't know the details yet, but I believe that some people asked for their personal data and what facebook does with it. It turns out that most data is kept even when you as a user removed it. It also turns out that it is not clear at all what facebook does with your data. And you have to keep checking the settings, because they change with every new service. Anyway...something definately worth looking into. (this was also mentioned by Mireilla btw)
% I think that from the case we can build towards our advice aka memorandom.

%With the previous assignment I got some feedback that, by quoting legal text you can make your argument stronger. Dunno if you knew about that or did it already...just thought i'd share :)
%We should also look into those SSRN articles on blackboard.

%%% REFS %%%

\bibliographystyle{plain} % amsalpha
\bibliography{assignment2}

\end{document}
