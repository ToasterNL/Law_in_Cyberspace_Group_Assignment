\documentclass[11pt]{article}
\usepackage{a4wide, graphicx, fancyhdr, wrapfig, tabularx, amsmath, amssymb, hyperref, color, verbatim, nameref}
\usepackage[english]{babel}
\definecolor{linkcolour}{rgb}{0,0.2,0.6}
\hypersetup{colorlinks,breaklinks,urlcolor=linkcolour, linkcolor=linkcolour}

%----------------------- Macros and Definitions --------------------------

\setlength\headheight{20pt}\usepackage{}
\addtolength\topmargin{-10pt}
%\addtolength\footskip{20pt}

\fancypagestyle{plain}{%
\fancyhf{}
\fancyfoot[RO,LE]{\sffamily\bfseries\thepage}
\renewcommand{\headrulewidth}{0pt}
\renewcommand{\footrulewidth}{0pt}
}

\pagestyle{fancy}
\fancyhf{}
\fancyfoot[RO,LE]{\sffamily\bfseries\thepage}
\fancyhead[RO,LE]{\textsc{}}
\fancyhead[LO,RE]{\emph{}}
\renewcommand{\headrulewidth}{1pt}
\renewcommand{\footrulewidth}{0pt}
\newcommand{\tab}{\hspace*{2em}}

\newcommand{\tocheck}[1]{{\bf !?: #1 :!?}}
\newcommand{\OBA}{Online behavioural advertising }
\newcommand{\oba}{online behavioural advertising }
\newcommand{\ePD}{ePrivacy Directive}

\frenchspacing

%-------------------------------- Title ----------------------------------

\title{\textbf{Law in Cyberspace\\ \emph{Behavioural Advertising}}}
\author{
	H.J. Tilmans
	\and M. Vijfvinkel
	\and Y. Zeng
	\and
	\href{mailto:h.j.tilmans@student.tue.nl}{h.j.tilmans}@student.tue.nl,\\
	\{\href{mailto:m.vijfvinkel@student.ru.nl}{m.vijfvinkel},
	\href{mailto:y.zeg@student.ru.nl}{y.zeng}\}@student.ru.nl
}
\date{\today}

\begin{document}
\maketitle
\section{Introduction}
The online advertising market topped a record revenue of 26 billion US Dollar over the year 2010. \cite{IAB2011}
That given, it is not surprising that the advertising business is constantly looking for methods to improve ad responses and thus revenue.
Nowadays, \oba - a technique that tracks online user's behaviour - is the main method used for delivering advertisements to each user's individual estimated interest. Estimated, because the delivered ads are chosen from a large database (containing advertisers' ads) according to the digital profile that is created for each online user. Clearly, that profile may not correspond with the real user's 'profile'.
Moreover, what exactly does the advertising world know about you?

In the next section, the terminology used in this \tocheck{paper} is explained. Then, in section~\nameref{techniques}, a short overview is given on the techniques used for \oba. In section~\nameref{legal implications}, the privacy violations \tocheck{and data protection violations} are defined. Then the current applicable laws from the EU and the European Convention of Human Rights are stated, followed by the section~\nameref{memorandum} in which advises to the courts, legislators and industries are given.

\section{Terminology}
\label{terminology}
Basically, in an \oba system, the following four main participating entities can be distinguished:

\begin{itemize}
	\item Ad network\\
			The ad network makes money by charging the advertisers who would like to use the big advertising infrastructure of the ad networks (e.g. Google's Adsense). With the use of online behavioural tracking, the ad networks can deliver ads to online users more accurately.

	\item Advertiser\\
			The advertisers use ad networks for spreading their ads more efficiently. Meaning, when using ad networks, the advertiser's ads most likely will be shown to interested users more than to uninterested users, which clearly should result in more revenue for the advertisers.

	\item Ad publisher\\
			The ad publisher shows the targeted ads to the online users through their website. In return for the used online web-space, the ad publisher receives money per click/view.

	\item Online user\\
			Browsing through the behavioural advertising supported websites, the user will receive more interesting, thus less annoying advertisements. Furthermore, thanks to the financial support of the \oba to publishers, the user can enjoy many more free online services, e.g. Google Mail. 
\end{itemize}

Clearly, everyone benefits from this \oba system.
When the ad networks use better tracking techniques, they will be able to target the ads better, which will have more users view/click on interesting ads and thus having the ad publishers earning more money, as well as the advertisers who will have more potential buyers on their website.
So, which techniques are used?

\section{\OBA techniques}
\label{techniques}
Online behavioural advertising completely relies on the creation of accurate user profile(s) by the ad networks. Whenever a user profile is incomplete or incorrect, the user will receive ads which he may not like. To avoid this, ad networks use, inter alia, the following techniques.

\begin{itemize}
	\item Cookies\\
		A cookie is a piece of text received from the web server - when a user is visiting a website - and is saved on his computer by the web browser. Aside from a unique user identifier, site preferences, authentication information, session identifiers and more can be stored in this cookie as well. So, whenever visiting the same website again, this cookie is sent to the website again which will adjust the content of the website according to your preferences and profile. Google Adsense, for example, stores a cookie with a unique number in order to be able to track you when visiting ad publisher's websites (which could be connected to Google Adsense's network). \cite{Adsense2011}

	\item Web Bugs\\
		A web bug usually is a small, seemingly invisible, part of a web page or non-plain text e-mail, which allows the ad network to track who has viewed the web page or e-mail. This web bug is downloaded from the ad networks server every time the web page or e-mail is opened, so the ad networks know when and what you opened. Possibly, JavaScript is used to extract even more information from you to ensure the ad networks use the right profile. For this technique, no cookies are required.

	\item Web browser fingerprinting\\
		A web browser fingerprint is the collection of all information about the browser version, operating system it runs on, installed plug-ins and their version numbers and more. In a publication of the Electronic Frontier Foundation (EFF), a web browser can already be uniquely identifiable among over 500,000 (privacy conscious) users. \cite{EFF2010} Like with the web bugs, the web browser fingerprinting does not require client-side storage. But, not all information can be extracted from the web browser without the use of JavaScript.

	\item Keystroke statistical analysis\\
		When knowing keystroke speeds of a user, one could identify a user according to their 'keystroke profile'. This technique is based on the fact that each user has a unique keystroke \tocheck{speed}. Google may be using the technique already, since whenever you have \textit{Google Instant} activated, Google receives every character you type in the search box separately. \cite{GoogleInstant} This technique requires JavaScript. 
\end{itemize}

As can be seen, lots of information can be gathered using these techniques. But, what privacy implications do these techniques introduce? Is it allowed to create digital profiles of users, without the user knowing? The legal implications and violations will be discussed in the next section.

\section{The legal implications}
\label{legal implications}

\section{Applicable laws}
\label{applicable laws}

To determine if the privacy of users is being violated, we first determined which European legislation applies to this problem and below we describe which articles of that leglislation apply.

\subsection{\ePD 2002/58/EC}

Out of the \ePD we use the following articles:

\begin{itemize}
	\item Definition Electronic Mail\\
		\emph{ ‘electronic mail’ means any text, voice, sound or image message sent over a public communications network which can be stored in the network or in the recipient's terminal equipment until it is collected by the recipient;}

	\item Article 3\\
		\emph{This Directive shall apply to the processing of personal data in connection with the provision of publicly available electronic communications services in public communications networks in the Community, including public communications networks supporting data collection and identification devices.}

	\item Article 6\\
		\emph{For the purpose of marketing electronic communications services or for the provision of value added services, the provider of a publicly available electronic communications service may process the data referred to in paragraph 1 to the extent and for the duration necessary for such services or marketing, if the subscriber or user to whom the data relate has given his or her prior consent. Users or subscribers shall be given the possibility to withdraw their consent for the processing of traffic data at any time.}

	\item Article 12\\
		\emph{Member States shall ensure that subscribers are informed, free of charge and before they are included in the directory, about the purpose (s) of a printed or electronic directory of subscribers available to the public or obtainable through directory enquiry services, in which their personal data can be included and of any further usage possibilities based on search functions embedded in electronic versions of the directory.}

	\item Article 13\\
		\emph{The use of automated calling and communication systems without human intervention (automatic calling machines), facsimile machines (fax) or electronic mail for the purposes of direct marketing may be allowed only in respect of subscribers or users who have given their prior consent.}

		\emph{Member States shall take appropriate measures to ensure that unsolicited communications for the purposes of direct marketing, in cases other than those referred to in paragraphs 1 and 2, are not allowed either without the consent of the subscribers or users concerned or in respect of subscribers or users who do not wish to receive these communications, the choice between these options to be determined by national legislation, taking into account that both options must be free of charge for the subscriber or user.}

\end{itemize}


\subsection{Legal implications of \ePD}

The articles mentioned above do not show all the subarticles, but show the main points related to this problem. The \ePD shows that if electronic communication services and public communications networks want to process personal data, they have to ask prior consent from the user to do so. The \ePD also shows that unsolicited communications used for direct marketing needs to have prior consent of users or national legislation has to make sure that the users wishes not to receive the communication are respected.



\subsection{Data Protection}

TODOnext



%---------------------------------------------

% So far I have read about half of the e-privacy directive and marked some text, that might be relevant. This is not the text with the actual articles yet, but the introductory text, marked by (number) on pages 1-10.

% (16), (17), (22), (24), (25), (26), the rest i still have to read.
% Another EU directive, is de data protection act (i believe it is 95/46/EC and correct if i'm wrong)...the eprivacy directive refers to that one.
% Directive 2009/136/EC refers to cookies in particular (dutch cookie law is based on this), which might be interesting to mention a way that law slows the advertisers down. We can use this as an example of something that is mostly ignored by websites so far/ might cause inconvience for users (accepting cookies al the time) and reduce revenue for advertisers. 
%From there we can build to the Europe vs Facebook case, of which I don't know the details yet, but I believe that some people asked for their personal data and what facebook does with it. It turns out that most data is kept even when you as a user removed it. It also turns out that it is not clear at all what facebook does with your data. And you have to keep checking the settings, because they change with every new service. Anyway...something definately worth looking into. (this was also mentioned by Mireilla btw)
% I think that from the case we can build towards our advice aka memorandom.

%With the previous assignment I got some feedback that, by quoting legal text you can make your argument stronger. Dunno if you knew about that or did it already...just thought i'd share :)
%We should also look into those SSRN articles on blackboard.




\section{Memorandum}
\label{memorandum}
\subsection{To the courts}
\subsection{To the legislators}
\subsection{To the industry}

\section{Conclusion}

%%% REFS %%%

\bibliographystyle{plain} % amsalpha
\bibliography{assignment2}

\end{document}
